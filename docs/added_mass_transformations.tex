\documentclass[onecolumn]{article}

\usepackage{amsmath}
\usepackage{bm}
\usepackage{hyperref}

\title{Added Mass Transformations}
\author{Joan Aguilar Mayans}
\date{\today}

\begin{document}
\maketitle

Given a body with added mass, we compute the kinetic energy as
\begin{equation}
    T = \frac{1}{2} \left( \bm{v}_P^T, \bm{\omega}^T \right) \left( \bm{M}_P + \bm{A}_P \right)
    \left(
    \begin{array}{c}
        \bm{v}_P \\
        \bm{\omega}
    \end{array}
    \right),
\end{equation}
where $\bm{v}_P$ is the linear velocity at some point $P$, $\bm{\omega}$ is the angular velocity, $\bm{M}_P$ is the spatial inertia matrix at point $P$, and $\bm{A}_P$ is the added mass matrix at point $P$; with all quantities expressed in an inertial frame.

Under the assumption that kinetic energy is invariant to the point chosen to compute it, even in the presence of added mass, we have
\begin{equation}
    T = \frac{1}{2} \left( \bm{v}_P^T, \bm{\omega}^T \right) \left( \bm{M}_P + \bm{A}_P \right)
    \left(
    \begin{array}{c}
        \bm{v}_P \\
        \bm{\omega}
    \end{array}
    \right)
    = \frac{1}{2} \left( \bm{v}_Q^T, \bm{\omega}^T \right) \left( \bm{M}_Q + \bm{A}_Q \right)
    \left(
    \begin{array}{c}
        \bm{v}_Q \\
        \bm{\omega}
    \end{array}
    \right),
\end{equation}
where $Q$ is just some other point.

We know that for the case without added mass, kinetic energy is invariant to the chosen point, therefore,
\begin{equation}
    \frac{1}{2} \left( \bm{v}_P^T, \bm{\omega}^T \right) \bm{M}_P
    \left(
    \begin{array}{c}
        \bm{v}_P \\
        \bm{\omega}
    \end{array}
    \right)
    = \frac{1}{2} \left( \bm{v}_Q^T, \bm{\omega}^T \right) \bm{M}_Q
    \left(
    \begin{array}{c}
        \bm{v}_Q \\
        \bm{\omega}
    \end{array}
    \right),
\end{equation}
which implies
\begin{equation}
    \frac{1}{2} \left( \bm{v}_P^T, \bm{\omega}^T \right) \bm{A}_P
    \left(
    \begin{array}{c}
        \bm{v}_P \\
        \bm{\omega}
    \end{array}
    \right)
    = \frac{1}{2} \left( \bm{v}_Q^T, \bm{\omega}^T \right) \bm{A}_Q
    \left(
    \begin{array}{c}
        \bm{v}_Q \\
        \bm{\omega}
    \end{array}
    \right).
    \label{eq:kinetic-added-mass}
\end{equation}

Using the kinematic properties of the rigid body, the linear velocity of point $Q$, can be expressed as
\begin{equation}
    \bm{v}_Q = \bm{v}_P - \hat{\bm{r}} \bm{\omega};
    \label{eq:pq}
\end{equation}
where $\bm{r} = \overline{PQ}$, and the operator $\hat{\,}$ produces the left cross-product matrix for the vector it is applied to.

Replacing \eqref{eq:pq} in \eqref{eq:kinetic-added-mass} we get
\begin{equation}
    \frac{1}{2} \left( \bm{v}_P^T, \bm{\omega}^T \right) \bm{A}_P
    \left(
    \begin{array}{c}
        \bm{v}_P \\
        \bm{\omega}
    \end{array}
    \right)
    = \frac{1}{2} \left( \bm{v}_P^T + \bm{\omega}^T \hat{\bm{r}}, \bm{\omega}^T \right) \bm{A}_Q
    \left(
    \begin{array}{c}
        \bm{v}_P - \hat{\bm{r}} \bm{\omega} \\
        \bm{\omega}
    \end{array}
    \right),
    \label{eq:kinetic-added-mass-2}
\end{equation}
where we have used the fact that $\hat{\bm{r}}^T = -\hat{\bm{r}}$.

We split the added inertia matrices in $3 \times 3$ blocks such that
\begin{equation}
    \bm{A}_P =
    \left[
    \begin{array}{cc}
        \bm{A}_{P11} & \bm{A}_{P12} \\
        \bm{A}_{P21} & \bm{A}_{P22}
    \end{array}
    \right]
\end{equation}
and
\begin{equation}
    \bm{A}_Q =
    \left[
    \begin{array}{cc}
        \bm{A}_{Q11} & \bm{A}_{Q12} \\
        \bm{A}_{Q21} & \bm{A}_{Q22}
    \end{array}
    \right].
\end{equation}
Expanding the left and right hand side of \eqref{eq:kinetic-added-mass-2} we get
\begin{equation}
   \frac{1}{2} \left( \bm{v}_P^T \bm{A}_{P11} \bm{v}_P + \bm{v}_P^T \bm{A}_{P12} \bm{\omega} + \bm{\omega}^T \bm{A}_{P21} \bm{v}_P + \bm{\omega}^T \bm{A}_{P22} \bm{\omega} \right)
   \label{eq:left-side}
\end{equation}
and
\begin{multline}
    \frac{1}{2} \left( \bm{v}_P^T \bm{A}_{Q11} \bm{v}_P + \bm{v}_P^T \left( -\bm{A}_{Q11} \hat{\bm{r}} + \bm{A}_{Q12} \right) \bm{\omega} + \bm{\omega}^T \left( \hat{\bm{r}} \bm{A}_{Q11} + \bm{A}_{Q21} \right) \bm{v}_P + \right. \\
    \left. + \bm{\omega}^T \left( -\hat{\bm{r}} \bm{A}_{Q11} \hat{\bm{r}} + \hat{\bm{r}} \bm{A}_{Q12} - \bm{A}_{Q21} \hat{\bm{r}} + \bm{A}_{Q22} \right) \bm{\omega} \right),
   \label{eq:right-side}
\end{multline}
respectively.

By matching terms between \eqref{eq:left-side} and \eqref{eq:right-side} we get the equality
\begin{equation}
    \bm{A}_P =
    \left[
    \begin{array}{cc}
        \bm{A}_{Q11} & -\bm{A}_{Q11} \hat{\bm{r}} + \bm{A}_{Q12} \\
        \hat{\bm{r}} \bm{A}_{Q11} + \bm{A}_{Q21} & -\hat{\bm{r}} \bm{A}_{Q11} \hat{\bm{r}} + \hat{\bm{r}} \bm{A}_{Q12} - \bm{A}_{Q21} \hat{\bm{r}} + \bm{A}_{Q22}
    \end{array}
    \right],
    \label{eq:transform1}
\end{equation}
or equivalently,
\begin{equation}
    \bm{A}_P = \bm{A}_Q +
    \left[
    \begin{array}{cc}
        \bm{0} & -\bm{A}_{Q11} \hat{\bm{r}} \\
        \hat{\bm{r}} \bm{A}_{Q11} & -\hat{\bm{r}} \bm{A}_{Q11} \hat{\bm{r}} + \hat{\bm{r}} \bm{A}_{Q12} - \bm{A}_{Q21} \hat{\bm{r}}
    \end{array}
    \right];
    \label{eq:transform2}
\end{equation}
which can be used to transform an added mass matrix between two arbitrary points.
Note that \eqref{eq:transform1} and \eqref{eq:transform2} are a generalization of the parameterization of a rigid-body inertia matrix \cite[eq.~(2.91)]{fossen1994}.



\begin{thebibliography}{9}
\bibitem{fossen1994}
    Thor I. Fossen,
    \textit{Guidance and Control of Ocean Vehicles},
    John Wiley \& Sons, England,
    1994.
\end{thebibliography}

\end{document}
